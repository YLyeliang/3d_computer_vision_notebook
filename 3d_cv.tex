\documentclass{article}

\usepackage[UTF8]{ctex}
\usepackage{amsmath}
%\usepackage{steinmetz}
\usepackage{graphicx}
\usepackage{geometry}
\usepackage[colorlinks,linkcolor=blue]{hyperref}
\geometry{a4paper,scale=0.75}
% left=2cm,right=2cm,top=1cm,bottom=1cm
\title{3D感知Notebook}
\author{叶亮}
\date{\today}
\begin{document} 
\maketitle
\tableofcontents

\section{静态感知-车道线}
车道线感知,根据网络结构的定义可以分成4类方法:1. segmentation-based. 2. anchor-based.  3. row-wise 4. parametric prediction.
\subsection{Segmentation-based Methods}
基于分割的方法,经典方法有:

\subsection{Anchor-based Methods}

\subsection{Row-wise Methods}

\subsection{Parametric Prediction Methods}


\end{document}
